\documentclass[12pt]{article}
\usepackage{graphicx}
\usepackage{hyperref}
\usepackage{amsmath}
%\usepackage{siunitx}
\usepackage{subcaption}
%\usepackage{amssymb}
\usepackage{float}
\usepackage{wrapfig}
\usepackage{caption}
\usepackage{etoolbox}
\usepackage{enumitem}

\makeatletter
\preto{\@verbatim}{\topsep=0pt \partopsep=0pt }
\makeatother
\date{}

\begin{document}
\begin{titlepage}
\begin{center}
\Huge
\textbf{User Guide for \\
AMMCR: Ab-initio model for Mobility and Conductivity Calculation by Using Rode Algorithm\\ }
\end{center}
\end{titlepage}
  
\tableofcontents

\newpage
  
\section{Introduction}

This code is to calculate mobility and conductivity of semiconductors by using ab-initio band structure and inputs and it is based on Rode \cite{rode1,rode2,rode3, rode4} algorithm. In this code, ionized impurity, polar optical phonon scattering due to longitudinal phonon, acoustic deformation scattering, piezoelectric scattering, dislocation scattering, alloy scattering, intravalley scattering and neutral impurity scattering are included. Out of these eight scattering mechanisms, any scattering mechanism can be included or excluded in the simulation.

\section{Getting Started}

Steps to run the above code are follows

\begin{enumerate}

\item Unzip the 'AMMCR.tar' file. It will unzip a folder 'AMMCR'. The AMMCR folder has four folders, 'code', 'utility', 'Examples' and 'manual'. 

\item 'code' folder contains the code to calculate mobility and conductivity. 

\item 'utility' folder contains a file 'k\_point\_generator.cpp', this file has a program that is used to generate a 'k\_points\_file', that will give the k-point file required for DFT simulation of the band structure. This k-point file should be used while doing the DFT calculation of band structure for materials.   

\item 'Examples' folder contains two folders, 'ZnSe' and 'CdS'. These two folders contain files required to test code for these two materials ZnSe and CdS. How to simulate the code with these input files for ZnSe and CdS is explained in section \ref{examples}.  

\item There are five input files input.dat, EIGENVAL, OUTCAR, DOSCAR and PROCAR are used by the program out of which three files input.dat, EIGENVAL and PROCAR are essential. So these files are first to be generated to run the code. EIGENVAL, OUTCAR, DOSCAR and PROCAR are generated by using the VASP program\cite{ref4,ref5,ref6}. input.dat file contains values of different constant required for calculation. Format and form of input.dat file is explained later in section \ref{input_files} in this manual.  

\item The distribution included a makefile for simulating the program by using gnu gcc compiler.  To compile the program, open the terminal in the folder 'code' and run the command \textbf{make}. It will compile different .cpp file inside the folder. It will generate an executable file AMMCR.

\item To simulate the program, run command \textbf{./AMMCR} at the terminal. It will run the program and generate different output files. Description of input and output files are explained in the next section. 

\end{enumerate} 
 
\section{Input Files} \label{input_files}
\begin{enumerate}
\item EIGENVAL  
\newline This file is generated by using the VASP program. The user has to use a
dense k-mesh around CBM. It contains band structure of the material. The k-point file required for this EIGENVAL is generated by using a program named kpoint\_generator.cpp included with this package. The program kpoint\_generator.cpp  will asks a center point around which a dense k points mesh is generated. This kpoint file should be used for band structure calculation. 
\item OUTCAR
\newline This file is also generated by using the VASP program. Some parameters lattice matrix, ion masses, the volume of the cell are taken from this file.
\item DOSCAR
\newline This file is also generated by using the VASP program while doing abinito band structure calculation. This file is not necessary for simulation. If DOSCAR is not given during simulation, then the program will run with free electron density of states. 
\item PROCAR
\newline This file also generated by using the VASP program. It will contains projection of conduction band over s and p type orbitals. If PROCAR is not given then the program will assume that the conduction band is purely s-like. 
\item input.dat
\newline In input.dat file contains different constants value required for simulation. All data values should be saved in a particular order. 
For example, a typical input.dat file is shown below 
\newline
\newpage ------------------------------------------------------------------------
\begin{quote}
\begin{verbatim}
200
3e15 3e16 3e17
2e12 6e15 8e16
3e11 4e12 7e14
7.45
3.45
0
1
1
0
1 1 1 1 1 0 1 1    
100
5.88
12             
0.0392
9.578e+11 
2.916e+11
0.53
0.177504328 	
0.47
7.96 8.34
3 4 
3 5
0
10
5
0.2 0.4 0.6 0.9
0.5
\end{verbatim}
\end{quote}
------------------------------------------------------------------------
\newline It contains data in following order and format 
\begin{enumerate}[label=\Roman*]

\item First line contains different temperatures at which calculation is done. Unit of temperature is Kelvin here. If the simulation is to be done at ten different temperatures, then ten different temperature values are given in the same line. For example, 
\newline 50 80 100 150 200 250 300 350 400 450
\newline There should a space between two temperature values.

\item The second line contains different donor doping concentrations. In the above example, three different donor doping concentrations are written 3e15 3e16 3e17. All doping concentrations are written in a line with a space between two values. Units of doping concentration is per $cm^3$.

\item The third line contains different acceptor doping concentrations. In the above example, three different acceptor doping concentrations are written 2e12 6e15 8e16. All doping concentrations are written in line with a space between two values. Unit of doping concentration is per $cm^3$. 

\textbf{NOTE} The length of donor and acceptor array concentration should be the same; for example, if the second line contains three doping concentration values, then the third should have three acceptor concentrations. The first element of the second line corresponds to the first element of the third line and so on.

\item The fourth line contains neutral impurity concentrations. In the above example, three different neutral impurity concentrations are written one for each three different doping concentrations. Unit of neutral impurity concentration is per $cm^3$. Both neutral impurity and donor or acceptor doping concentration array must be of the same length. In the above example, there are three donor or acceptor doping concentrations and three neutral impurity concentrations. For donor doping concentration 3e15, neutral impurity concentration is 3e15, for donor doping 3e16 neutral impurity is 4e12 and for donor doping 3e17, neutral impurity is 7e14. For the first element of donor or acceptor doping concentration, the first element of neutral impurity, for the second element of donor or acceptor doping the second element of neutral impurity is considered.
\newline \textbf{NOTE:-} Out of temperature and doping array, one must be of length one. If temperature and doping array both are greater than one length, then the program would stop. For example, in the above shown file temperature array length is one and donor or acceptor doping array length is three. 

\item The fifth line contains the value of the relative low-frequency dielectric constant. In the above example, the relative low-frequency dielectric constant is 7.45.  

\item The sixth line contains the value of the relative high frequency dielectric constant. In the above example, the relative high frequency dielectric constant is 3.45.

\item The seventh line contains the energy bandgap of the semiconductor. It's unit is eV. Suppose the energy band gap is given as 0 as in the above example. In that case, the program will calculate the energy bandgap from the difference between conduction band minima and valence band maxima. If some value is given in place of 0, then that value is used for calculation.


\item The eighth line contains the equivalent number of valence band valleys within the Brillouin zone. 

\item The ninth line contains the equivalent number of conduction band valleys within the Brillouin zone. For example, in the case of silicon, conduction band minima lie along (100) directions. So, there are six equivalent (100) directions. So there six equivalent conduction band valleys within the Brillouin zone.

%\item Tenth line contains lattice constant of material used. Its unit is nanometre.

\item The tenth line contains the density of semiconductor. Its unit is $gm/cm^3$. If the density is given as zero than, this program will automatically calculate density using OUTCAR file data. If some value is given than, that value is used for calculation.

\item The eleventh line contains values that control different scattering mechanisms that will be included in the simulation. Every element of this line is used to switch on or off a particular scattering mechanism. If an element is zero, then corresponding scattering mechanism is not included in the scattering mechanism and if an element is one then, that particular scattering mechanism is included in the simulation. The element number of the line with the corresponding scattering mechanism is written below. 
	\begin{enumerate}
	\item Ionized impurity scattering 
	\item Polar Optical phonon scattering due to longitudinal phonon
	\item Acoustic deformation scattering
	\item Piezoelectric scattering
	\item Dislocation scattering
	\item Alloy scattering
	\item Intra-valley scattering
	\item Neutral impurity scattering
	\end{enumerate}
\item The twelfth line contains dislocation density. Dislocation density is measured in units of per $cm^2$. The expression used for dislocation scattering is applies to only wurtzite structure. For calculation, overall density is required. We assumed that dislocation density is homogeneously distributed along c-axis of the wurtzite structure. So, the given dislocation density is divided by lattice constant along the c-axis for overall density.

\item The thirteenth line contains longitudinal polar optical phonon frequency. Its unit is THz here.
\item The fourteenth line contains acoustic deformation potential. Its unit is eV. 
\item The fifteenth line contains the piezoelectric coefficient. It is dimensionless. It is calculated by using the equation\ref{pzcoeff_s} for sphalerite crystal structure \cite{rode1}

\begin{equation}
\ P^2 = h_{14}^2\frac{[(\frac{12}{c_l})+(\frac{16}{c_t})]}{35} 
\label{pzcoeff_s}
\end{equation}
where $h_{14}$ is one element of piezoelectric stress tensor and  $c_l$, $c_t$ are the spherically averaged elastic constant for longitudinal and transverse modes respectively.

For wurtzite crystal structure, the piezoelectric coefficient P can be calculated by using equations \cite{ref1}
\begin{equation}
\ P_{\perp}^2 = 4 \epsilon_0 \frac{(21 h_{15}^2 + 6 h_{15} h_{x } + h_x^2)}{105c_t} + \epsilon_0 \frac{(21 h_{33}^2 - 24 h_{33} h_x + 8 h_x^2)}{105c_l}  
\label{pzcoeff_w1}
\end{equation}

\begin{equation}
\ P_{\parallel}^2 = 2 \epsilon_0 \frac{(21 h_{15}^2 + 18 h_{15} h_{x } + 5 h_x^2)}{105c_t} + \epsilon_0 \frac{(63 h_{33}^2 - 36 h_{33} h_x + 8 h_x^2)}{105c_l}  
\label{pzcoeff_w2}
\end{equation}

\begin{equation}
\ h_x = h_{33} - h_{31} - 2h_{15}
\label{hx}
\end{equation}
where $h_{15}$, $h_{31}$ and $h_{33}$ are the three independent elements of the piezoelectric stress tensor of wurtzite structure.
For wurtzite crystal structure, there are two piezoelectric coefficients $P_{\parallel}$ and $P_{\perp}$ for drift mobility measured parallel or perpendicular to the c-axis of the crystal. 

\item The sixteenth line contains the spherically averaged elastic constant for longitudinal modes. Its unit is $dyne/cm^2$.  
For sphalerite crystal structure, it is calculated by equation \ref{cl_s}

\begin{equation}
\ c_l = (3c_{11} + 2c_{12} + 4c_{44})/5 
\label{cl_s}
\end{equation}
where  $c_{11}$, $c_{12}$ and $c_{44}$ are elastic constants.

For wurtzite crystal structure, it can be calculated by equation \ref{cl_w}
\begin{equation}
\ c_l = (8c_{11} + 4c_{13} + 3c_{33} + 8c_{44})/15 
\label{cl_w}
\end{equation}
where  $c_{11}$, $c_{13}$, $c_{33}$ and $c_{44}$ are elastic constants.


\item The seventeenth line contains the spherically averaged elastic constant for transverse modes. Its unit is $dyne/cm^2$. For sphalerite crystal structure, it is given by equation \ref{ct}

\begin{equation}
\ c_t = (c_{11} - c_{12} + 3c_{44})/5 
\label{ct}
\end{equation}
where $c_{11}$, $c_{12}$ and $c_{44}$ are elastic constants.
For wurtzite crystal structure, it can be calculated by equation \ref{ct_w}
\begin{equation}
\ c_t = (2c_{11} - 4c_{13} + 2c_{33} + 7c_{44})/15 
\label{ct_w}
\end{equation}
where $c_{11}$, $c_{13}$, $c_{33}$ and $c_{44}$ are elastic constants.

%\item Eighteenth line contains piezoelectric stress tensor $e_{14}$. It is measured in %units of $C/m^2$ ..................... doubt.....equation
	
\item The eighteenth contains alloy potential. Its unit is eV. 

\item The nineteenth contains the volume of the primitive cell of the alloy. Its unit is $(nm)^3$. 

\item The twentieth contains the fraction of the atom for alloy. 
\newline \textbf{NOTE:} Line number eighteenth, nineteenth and twentieth data required for alloy scattering only. If alloy scattering is not required while simulation or kept off, any value of these three parameter can be given. It does not affect simulation results.
 
\item Line twenty-one contains phonon frequency for intravalley scattering. Its unit is here THz. If it is required to run code for two different intravlley scattering, then two frequency values are given.  

\item Line twenty-two contains  the value of the coupling constant for intravalley scattering. Its unit is here $10^8$ eV/cm. If two values of phonon frequency for intravalley scatering is given then, two values of coupling constant are required. 

\item The line twenty-three contains the number of the final valley for scattering. 
If we want to run simulation for two different intravalley phonon frequencies then we have to specify two sets of frequency, two sets of coupling constant and two sets of the final valley.
\newline For example, in the above given input.dat file, two sets of phonon frequency, coupling constants and the number of final valleys are given means program will simulate with these two sets for intra-valley scattering.   
\newline \textbf{NOTE:} Data of line number twenty-one, twenty-two and twenty-three are required for intravalley scattering only. If intravalley scattering is kept off then any value can be used in these three lines, it does not affect results. 

\item Line twenty-four specifies that for density of states, whether DOSCAR is used or free electron density is used for calculation. If it is 0, then DOSCAR is used for the density of states calculation or if it is 1, then free electron density of states is used for simulation. If DOSCAR is not given and then the program will automatically simulate with the free electron density of states.

\item Line twenty-five specifies the number of iteration for calculating perturbation in the distribution function. The convergence is exponential, so it requires only a few iterations. Typically six iterations are sufficient for convergence.

\item Line twenty-six specify the degree of polynomial to be used for fitting the conduction or valence band. Conduction band or valence band is used to obtain the velocity of carriers at different values of wave vector k. For getting velocity, the derivative of the band is to be calculated at different values of wave vector k. For getting the smooth curve of derivative, the conduction band or valence band is divided into three to four segments and each segment is fitted with a polynomial. If the degree of the polynomial is given 0, then by default program will automatically take it to be six degree.   

\item Line twenty-seven specifies the fraction at which the k segment is to be divided for analytical fitting. For example in the input.dat file shown above fraction is given 0.2 0.4 0.6 0.9, it means the program will first calculate the maximum value of wave vector k in the Brillouin zone and multiply these fraction with the maximum value of wave vector (let $k_{max}$), it means conduction or valence band is divided at the points \\ 

$0.2*k_{max}$ \\
$0.4*k_{max}$ \\
$0.6*k_{max}$ \\
$0.9*k_{max}$ \\

The program will take these points as the starting point for division; then, the program will automatically search points for minimum discontinuity around the given points for fitting the curve. These points will not be the final division of wave vector $k$. The final wave vector $k$ division is decided by searching the points of minimum discontinuity; these given points can be taken as the starting point for checking discontinuity. There will be only three or four points that can be given for division; otherwise, the program will stop execution. If the user gives its value as 0 only, the code will automatically calculate points for division with minimum discontinuity.       

\item Line twenty specifies the magnetic field applied. Its unit is Tesla. The magnetic field value is required to calculate the hall factor and hall mobility. If the user does not want to include the magnetic field effect, the magnetic value should be zero. For details of hall mobility and hall factor, refer to the paper \cite{rode4}.

\end{enumerate}
\textbf{Note:-} If some scattering mechanism is not included in the simulation, then some dummy value should be given for constants in the input.dat file that are required for the particular scattering. For example alloy potential, volume of the primitive cell and alloy fraction these three parameters are required for alloy scattering. If alloy scattering is not to be included in the scattering then, any dummy value is used for them, it does not affect simulation result.
\end{enumerate}
   
%\newpage
\section{Output Files} 
The program will generate the following different output files on simulation.
\begin{enumerate}

\item mobility.dat
\newline File 'mobility.dat' contains drift mobility of electron at different temperature or doping and also contains the contribution of mobility from different scattering mechanisms.

\item mobility\_hall.dat
\newline File 'mobility\_hall.dat' contains hall mobility of electron at different temperature or doping and also contains the contribution of hall mobility from different scattering mechanisms. For calculation of the hall mobility effect of the magnetic field is also included in the simulation.

\item conductivity.dat
\newline File 'conductivity.dat' shows conductivity calculated by using Rode and RTA (relaxation time approximation) method for different temperatures or doping concentrations. 

\item conductivity\_hall.dat 
\newline File 'conductivity\_hall.dat' shows conductivity calculated from hall mobility using Rode and RTA (relaxation time approximation) method for different temperatures or doping concentrations. 

\item themopower.dat
\newline File 'thermopower.dat' shows thermopower at different temperatures or doping concentrations.
  
\item scattering\_rate.dat
\newline File 'scattering\_rate.dat' contains the scattering rate of different scattering mechanisms included in the simulation as well as the total relaxation time obtained by applying the Matthiessen rule. It will show the scattering rate for only the last element of the temperature array. For example, we are doing a simulation for temperature array 120 150 180 200; then it will contain scattering rate for only the last element of temperature array at 200 K. 

\item g.dat
\newline File 'g.dat' contains perturbation in distribution function for different values of energy of conduction band. It will show perturbation in the distribution function for only the last element of the temperature array. This perturbation is obtained while doing simulation for drift mobility \cite{rode1}.

\item gH.dat or hH.dat
\newline File 'gH.dat' and 'hH.dat' contain perturbation in distribution function due to electric field and magnetic field for different energy values of the conduction band. These files will show perturbation in the distribution function for only the last element of the temperature array. These perturbations in distribution function are obtained while doing simulation for hall mobility \cite{rode4}.

\item hall\_factor.dat
\newline File 'hall\_factor.dat' shows the value of hall factor calculated using Rode and RTA method for different temperature or doping concentrations.

\item mean\_free\_path.dat
\newline File mean\_free\_path.dat specifies the value of mean free (in nm) at different values of energy.

\item conduction\_band.dat
\newline File 'conduction\_band.dat' specifies the values of energy (in eV) and wave vector k values (in 1/nm) for the conduction band.

\item valence\_band.dat
\newline File 'valence\_band.dat' specifies the values of energy (in eV) and wave vector k values (in 1/nm) for the valence band. 

\end{enumerate}

\textbf{Note:-} For spin-polarized calculations, the program will generate two sets of output files - one set for up - spin calculation and the other set for down spin calculations. 
  
  
\section{Installation and Prerequisites}
To simulate the code Unix/Linux operating system is required with GNU GCC compiler installed. The Program is tested with gcc version 5.4.0.
\newline To check whether gcc compiler is already installed or not installed on your operating system run the following command at the terminal:

\begin{quote}
\begin{verbatim}
gcc -v
\end{verbatim}
\end{quote}

If gcc compiler is not installed, then run following command at the terminal to install gcc compiler: 

\begin{quote}
\begin{verbatim}
sudo apt-get install gcc
\end{verbatim}
\end{quote}

\section{Examples} \label{examples}
\subsection{Example 1 - ZnSe}
'Examples' folder contains a folder 'ZnSe', this folder contains five files 'EIGENVAL', 'PROCAR', 'DOSCAR', 'OUTCAR' and 'input.dat'. These files contains ZnSe band structure, wave function admixture, the density of states and different parameters for ZnSe required for simulation \cite{paper1}. Copy these five files to folder 'code' to run code for ZnSe. Open the terminal in the folder and run command \textbf{./AMMCR} at the terminal. It will generates following output files 'mobility.dat', 'conductivity.dat', 'thermopower.dat', 'scattering\_rate.dat', 'g.dat', 'mean\_free\_path.dat', 'conduction\_band.dat' and 'valence\_band.dat'. The files 'mobility.dat', 'conductivity.dat' and 'thermopower.dat' will contain mobility, conductivity and  Seebeck coefficient at different temperatures respectively. 'scattering\_rate.dat' file contains scattering rate for different scattering mechanisms at only the last temperature point of temperature array. 'g.dat' file contains perturbation in distribution function at only last temperature point of temperature array. 'mean\_free\_path.dat' file contains mean free path (in nm) at different values of energy. 'conduction\_band.dat' and 'valence\_band.dat' files contain conduction band and valence band E-k values for conduction band and valence band respectively.
 
\subsection{Example 2 - CdS}
Similarly 'Examples' folder contains a folder 'CdS', this folder contains five files 'EIGENVAL', 'PROCAR', 'DOSCAR', 'OUTCAR' and 'input.dat' for CdS simulation. These files contains CdS band structure, wave function admixture, density of states and different parameters for CdS required for simulation \cite{paper2}. Copy these five files to folder 'code'. Open a terminal in the folder and run command \textbf{./AMMCR} at the terminal. It will generates eight output files 'mobility.dat','conductivity.dat','thermopower.dat', 'scattering\_rate.dat', 'g.dat', 'conduction\_band.dat', 'mean\_free\_path.dat' and 'valence\_band.dat'. 

%-------------------------------------------------------------------------------------------------------
\bibliographystyle{ieeetr}
\bibliography{Reference.bib}

\end{document}

